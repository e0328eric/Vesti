\documentclass[item,korean]{coprime}

\usepackage{tikz}
\usepackage[many]{tcolorbox}
\usepackage{enumitem}
\usepackage{array}
\usepackage[a4paper,margin=0.4\textwidth]{geometry}

\begin{document}
Now the main document is started! The math mode can be used with $1+1<2$.
Note that $3\leq 2$ will compile first by \$3\\leq 2\$.
\[
    \text{In here, I can type text}
\]

To use the latex function, type "LaTeX". If it has a parameter,
then type \textbf{like this!}. This grammar can work in the math mode.

Finally, to use an environment, there are two ways to do this:
First is that use begenv and endenv keywords. For example,
\begin{center}
    \begin{minipage}{0.7\textwidth}
        Use like this!
    \end{minipage}
\end{center}

Second way is use raw latex grammar. Upper part is equivalent with

\begin{center}
    \begin{minipage}{0.7\textwidth}
        Use like this!
    \end{minipage}
\end{center}


A token \#\#- and -\#\# is actually a long line vesti code verbatim.

\end{document}

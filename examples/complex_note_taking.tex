%
%  This file was generated by vesti 0.2.8
%
\documentclass[item,tikz]{coprime}

\settitle{Complex Analysis Rudin Note-taking}{Subgbae Jeong}

\def\arg{\mathop{\rm{arg}}}
\def\Arg{\mathop{\rm{Arg}}}
\def\Log{\mathop{\rm{Log}}}

\begin{document}
\fancyon
\section{Basic Theories of Holomorphic Functions}

\begin{defin}[Some short definitions]
\begin{itemize}
\item $-\pi<\Arg z\leq \pi$
\item $\Log z$ is a principle logarithm
\end{itemize}
\end{defin}

\begin{defin}[Branch of the Logarithm]
Let $G$ be a connected open set such that $0\notin G$. If there is a continuous function
$g:G\rightarrow \C$ such that $exp(g(z))=z$ for every $z\in G$, we call the function $g$ as a
\bf{branch of the logarithm}.
\end{defin}

We do not know the existence yet. However, if it exists, the branch of the logarithm is unique
under the modulo $2\pi i$ in the sense that the following theorem:

\begin{thm}
Let $G$ be an open connected subset of $\C$ and if $f$, $g$ are two branches of the logarithm
on $G$, then $f-g=2k\pi i$ for some $k\in\Z$.
\proof
Note that $\exp(f(z))=\exp(g(z))=z$ for all $z\in G$, thus
\[
    e^{f(z)-g(z)} = \frac{e^{f(z)}}{e^{g(z)}} = \frac{z}{z} = 1
\]
holds for all $z\in G$.
This means that if we let $h:=f-g$, then $h[G]\subset 2\pi i\Z$.
Since $h$ is continuous and $G$ is connected, $h[G]$ is also connected.
This guarantees a fixed integer $k$ such that $h(z)=2k\pi i$. \proved
\end{thm}

\begin{prop}\label{prop1}
Let $G$ and $\Omega$ are open subsets of $\C$.
Suppose that $f:G\rightarrow \C$ and $g:\Omega\rightarrow \C$ are continuous such that $f[G]\subset \Omega$ and
$g(f(z))=z$ for all $z\in G$.
If $g$ is holomorphic and $g'\neq 0$ on $\Omega$, then $f$ is also holomorphic, and
\[
    f'(z) = \frac{1}{g'(f(z))}.
\]
\proof
Fix $a\in G$ and $h\in\C$ such that $h\neq 0$ and $a+h\in G$.
Since
\[
1 = \frac{g(f(a+h))-g(f(a))}{h} = \frac{g(f(a+h))-g(f(a))}{f(a+h)-f(a)}\frac{f(a+h)-f(a)}{h}
\]
Here, as $g\circ f$ is identity, $f(a+h)$ cannot be equal to $f(a)$ since $h\neq 0$.
Taking $h\rightarrow 0$, the continuity of $f$ and differentiability of $g$ gives the result. \proved
\end{prop}

\begin{coro}
A branch of the logarithm is analytic and its derivative is $1/z$.
\proof If $f:G\rightarrow \C$ is a branch of the logarithm on some connected open set $G$,
as $\exp$ is holomorphic with nonzero derivative, appling \cref{prop1} gives the proof. \proved
\end{coro}

\begin{defin}
Let $f$ be a branch of the logarithm on an open connected set $G$, $b\in\C$ is fixed.
Define $g:G\rightarrow \C$ by $g(z)=\exp(bf(z))$.
If $n\in\Z$, then $g(z)=z^n$. In this manner, we define a branch of $z^b$ where $b\in\C$
for an open conencted set $G$ on which there is a branch of the logarithm,
and define $z^b:=\exp(bf(z))$.

If we write $z^b$ as a function, we will always understand that $z^b=\exp(b\Log z)$.
\end{defin}

\begin{thm}
Suppose $\mu$ is a complex measure on a measurable space $X$, $\phi:X\rightarrow \C$ is a measurable function,
$\Omega\subset\C$ is open such that $\Omega\cap\phi[X]=\emptyset$.

Define
\[
    f(z) := \int_X\frac{\mu(dx)}{\phi(x) - z}.
\]
Then $f$ is representable by a power series in $\Omega$.
\proof
Fix $a\in\Omega$ and $r>0$ such that $D(a;r)\subset\Omega$.
Since $\Omega$ and $\phi[X]$ are disjoint, $|\phi(x)-a|\geq r$.
This gives that
\[
    \frac{|z-a|}{|\phi(x) - a|} \leq  \frac{ |z-a|}{r} < 1
\]
for all $z\in D(a;r)$.
Thus, for all $z\in D(a;r)$, we get
\begin{align*}
\frac{1}{\phi(x) - z} &= \frac{1}{\phi(x) - a + a - z}\\
&= \frac{1}{(\phi(x)-a)\left(1+\frac{a-z}{\phi(x)-a}\right)}\\
&= \frac{1}{\phi(x)-a}\sum_{n=0}^\infty\frac{(z-a)^n}{(\phi(x)-a)^n},
\end{align*}
which the right side of the summation converges uniformly for all $x\in X$ for fixed $z$.
Hence,
\begin{align*}
f(z) &= \int_X\frac{\mu(dx)}{\phi(x) - z}\\
&= \sum_{n=0}^\infty\int_X\frac{(z-a)^n}{(\phi(x)-a)^{n+1}}\mu(dx)\\
&= \sum_{n=0}^\infty\left(\int_X\frac{\mu(dx)}{(\phi(x)-a)^{n+1}}\right)(z-a)^n
\end{align*}
so the proof is finished. \proved
\end{thm}

\subsection{Integration over Paths}

\begin{defin}[A curve]
Let $X$ be a topological space. A \bf{curve} in $X$ is a continuous function $\gamma:[a,b]\rightarrow X$ where
$[a,b]\subset\R$ with $a<b$.
The interval $[a,b]$ is called a \bf{parameter interval} of $\gamma$.
$\gamma^*$ is an image of $\gamma$, that is, $\gamma^*=\gamma[[a,b]]$.
$\gamma$ is called \bf{closed} if $\gamma(a)=\gamma(b)$.
\end{defin}

\begin{defin}[A path]
A \bf{path} is a picewise continuously differentiable curve in $\C$.
That means that there is a partition $P=\{a=t_0<t_1<\cdots<t_n=b\}$ for a parameter interval
$[a,b]$ of $\gamma$ such that $\gamma'$ exisis and continuous for all $(t_i,t_{i+1})$ and
$\gamma'(t_i+)$ and $\gamma'(t_i-)$ are exists and \it{finite}.
\end{defin}

\begin{defin}[Integration by Path]
Let $\gamma:[a,b]\rightarrow \C$ be a path and $f$ is a continuous on $\gamma^*$. Then we define
\[
    \int_\gamma f(z)dz := \int_a^b f(\gamma(x))\gamma'(x)dx
\]
\end{defin}

If $\phi:[a',b']\rightarrow [a,b]$ is a $C^1$ bijective function with $\phi(a')=a$ and $\phi(b')=b$,
Then we can think the new path $\gamma'=\gamma\circ\phi$. Note that $\gamma'^*=\gamma^*$.
As
\[
    \begin{aligned}
    \int_{\gamma'}f(z)dz = \int_{a'}^{b'}f(\gamma_1(t))\gamma_1'(t)dt
    &= \int_{a'}^{b'}f(\gamma(\phi(t)))\gamma'(\phi(t))\phi'(t)dt \\
    &= \int_{a}^{b}f(\gamma(t))\gamma'(t)dt = \int_\gamma f(z)dz,
    \end{aligned}
\]
the definition of the integration by path is well-defined.

\end{document}
